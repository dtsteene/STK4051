\documentclass[a4paper,12pt]{article}
\usepackage{amsmath, amsthm, amsfonts, amssymb}
\usepackage{mathtools}
\usepackage{microtype}
\usepackage{geometry}
\usepackage{booktabs}
\usepackage{graphicx}
\usepackage{tikz}
\usepackage{caption}
\usepackage{subcaption}
\usepackage{comment}

\usepackage{hyperref}
\usepackage{xcolor}
\hypersetup{ % this is just my personal choice, feel free to change things
    colorlinks,
    linkcolor={red!50!black},
    citecolor={blue!50!black},
    urlcolor={blue!80!black},
}

\colorlet{myred}{red!20}
\colorlet{myblue}{blue!20}
\colorlet{mypurple}{purple!40}
\colorlet{myorange}{orange!20}
\colorlet{myteal}{teal!35}

% 1. Define a ‘breaktheorem’ style that:
%    - Uses bold for the theorem heading,
%    - Puts (number + optional note) on the *same line*,
%    - Forces a line-break before the body text.
\newtheoremstyle{breaktheorem}%
{\topsep}{\topsep}%   % Above/below space
{
    \addtolength{\@totalleftmargin}{3.5em}
    \addtolength{\linewidth}{-3.5em}
    \parshape 1 3.5em \linewidth % chktex 1
    \itshape
}% body font
{}%                   % Indent
{\bfseries}%          % Head font
{.}%                  % Punctuation after theorem head
{\newline}%           % Space (or line break) after theorem head
{\thmname{#1} \textit{\thmnote{#3}}}
\makeatother
%   #1 = Theorem name ("Theorem")
%   #2 = Theorem number ("4.3")
%   #3 = The optional note (e.g., "Cea's lemma")

% 2. Tell amsthm to use this style for Theorem.
\theoremstyle{breaktheorem}
\newtheorem{theorem}{Theorem}[section]

\newtheoremstyle{exerciseStyle}
{ } % Space above
{ } % Space below
{\normalfont} % Body font
{ } % Indent amount
{\bfseries} % Theorem head font
{ } % Punctuation after theorem head
{ } % Space after theorem head
{\thmname{#1}} % Theorem hd spec (can be left empty, meaning `normal`)
\theoremstyle{exerciseStyle}
\newtheorem{exercise}{Exercise}[section]

\newtheoremstyle{solutionStyle}
{ } % Space above
{ } % Space below
{\normalfont} % Body font
{ } % Indent amount
{\bfseries} % Theorem head font
{ } % Punctuation after theorem head
{ } % Space after theorem head
{\thmname{#1}} % Theorem head spec (can be left empty, meaning `normal`)
\theoremstyle{solutionStyle}
\newtheorem{solution}{Solution}[section]


\title{Mandatory Assignment 1 STK4051}

\author{Daniel Steeneveldt}
\date{\today}

\begin{document}

\maketitle

\begin{exercise}
Recall that for a two-dimensional problem, the Nelder–Mead algorithm
maintains at each iteration a set of three possible solutions defining the
vertices of a simplex, specifically a triangle.
Let us consider whether three is a good choice. Imagine an algorithm
for two-dimensional optimization that maintains four points defining the
vertices of a convex quadrilateral and is similar to Nelder–Mead in spirit.
Speculate how such a procedure could proceed. Consider sketches like those
shown in Figure 2.10 of Givens $\&$ Hoeting (2013). What are some of the
inherent challenges?
\end{exercise}

\begin{solution}
    
    Nelder–Mead in 2D organizes its search using three vertices of a a triangle.
    With three vertices, the reflection, expansions, contractions, or shrink steps support a rough local approximation to a gradient step. 
 
    Below are a few plausible ways to define the reflection and centroid steps for four vertices,
     and the reasons each approach can fail or becomes difficult. 
    The conclusion is that quadrilaterals are more complicated than triangles.
    
    
    \subsection*{Naive Extension 1: Reflecting the Worst Point Across the Centroid of the Three Best Vertices}
    
    Imagine you have a symetric, convex quadrilateral—none of its interior angles poke inward.
    When you reflect one corner across the centroid of the other three we get the shape to bend inward - 
    it becomes non-convex. 

    Non-convex is bad, because we confuse “Inside” vs. “Outside”: When the quadrilateral folds, 
    it is no longer obvious which part of the region is “inside” the shape. 
    Expansion and shrinkage is simpler with a convex shape.
    A folded shape can make these steps contradictory or meaningless—reflections, expansions, or contractions might end up doing weird zig-zags because they rely on a neat polygonal boundary.

    
    If every reflection risks caving in the shape, you can see how repeated “folds” quickly lead to chaos.
     The algorithm might jump around and fail to converge in a sensible way.
    
    Nelder–Mead works smoothly with triangles (three points) because reflecting one corner across the centroid of the other two keeps the shape a triangle—there’s no extra corner to cause folds. With four vertices in 2D, it is far easier to mangle the shape in a single reflection if we don’t impose more rules or safeguards.
  
    \subsection*{Naive Extension 2: Reflecting the Worst Edge Across the Opposite Edge}
    
Instead of reflecting just the single worst vertex, one might try to reflect the whole worst edge—that is,
 the line segment connecting the two vertices with the highest function values. 
 For instance, label the quadrilateral’s vertices \(x_1\), \(x_2\), \(x_3\), \(x_4\) 
 so that \(x_3\) and \(x_4\) are the worst, and reflect that edge across the line by \(x_1\) and \(x_2\).  
Reflecting an edge across the opposite seems more similar to the triangle operation, since we keep
more of the original shape, and don't risk folding the quadrilateral in to a non-convex shape (I think).

In the triangle setup, the reflection step \emph{directly} sends the shape away from the worst vertex, towards the other two (better) ones. 
With four vertices, this mirroring does not move direcly away from the worst point.
This can lead to slower convergence, more cycling behavoiur and getting stuck in suboptimal regions.

To sumarize, the flexibility of the quadrilateral shape can make it harder to define a consistent reflection step that always pushes away from the worst vertex.
\end{solution}
    

\end{document}
